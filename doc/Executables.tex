%%%%%%%%%%%%%%%%%%%%%%%%%%%%%%%%%%%%%%%%%%%%%%%%%%%%%%%%%%%%%%%%%%%%%%%
% Developing under the Euro3D environment
% Set up your own project
%
% v1.0 07/03/2003 PF Creation from Yannick's developer guide
%
%%%%%%%%%%%%%%%%%%%%%%%%%%%%%%%%%%%%%%%%%%%%%%%%%%%%%%%%%%%%%%%%%%%%%%%
\label{S:Executables}

All executables are found in the {\tt user/bin} subdirectory.
Only some of the executables provided to the user are to be used in normal
preprocessing operations. The other executables include partial
preprocessing tasks, as well as analysis and miscellanous facilities.

\begin{subsection}{Main flow of preprocessing}

An ideal session of preprocessing could look like this
\begin{verbatim}
> preprocess_bias -in RAW_BIAS.cat -out BIAS_FRAME.cat -stackout\
   BIAS_FRAME.fits 
> preprocess_dark -in RAW_DARK_FRAME.cat -out PRE_DARK_FRAME.cat\
   -stackout PRE_DARK_FRAME.fits -bias BIAS_FRAME.fits
> preprocess_flat -in RAW_flat_FRAME.cat -out PRE_flat_FRAME.cat \
   -bias BIAS_FRAME.fits -dark PRE_DARK_FRAME.fits
> preprocess -in RAW_OBJ_FRAME.cat -out almost_PRE_OBJ_FRAME.cat\
   -bias BIAS_FRAME.fits -dark PRE_DARK_FRAME.fits -flat\
   PRE_flat_FRAME.fits 
\end{verbatim}
If you are only interested in basic preprocessing without bias, dark
of flat-field corrections, and do not need any variance frame, you can
directly use 
\begin{verbatim}
> preprocess -in RAW_OBJ_FRAME.cat -out almost_PRE_OBJ_FRAME.cat -fast
\end{verbatim}
By default, the preprocessing does not handle the guiding image. If
you want it glued together to the photometric image, you have to add a
-all option : 
\begin{verbatim}
> preprocess -in RAW_OBJ_FRAME.cat -out almost_PRE_OBJ_FRAME.cat -all
\end{verbatim}

These steps can then be completed,
for the photometric channel by
\begin{verbatim}
> split_image -in almost_PRE_OBJ_FRAME.fits -out PRE_OBJ_FRAME.fits
\end{verbatim}   
and for the red channel by 
\begin{verbatim}
> image_flip -in almost_PRE_OBJ_FRAME.cat -out PRE_OBJ_FRAME.cat -yflip
\end{verbatim}
(in order to get the image aligned with
the optical model used in the extraction)

In the previous description, 
'{\tt >}' refers to the prompt, and has not to be typed by the
user. 
The first name on the command line is the executable that can be found in the {\tt user/bin} package
subdirectory.
The rest of the line consists of program arguments. The {\tt -xx}
arguments can be placed in any order on the command line, but must be
followed by their non-dashed parameters (so this is the entire block
like {\tt -xx yy.fits} that may be moved). In addition to the specific
arguments listed above, there are all the IFU library standard
arguments available. The most useful are {\tt -h} 
for summary help
and {\tt -noask} if you don't want the prompt before overwriting an
existing file.
The
{\verb \  } 
indicates the continuation of the command line to the next one.
The file arguments are indicated as fits files ({\tt .fits}) or catalogs ({\tt .cat}).
A catalog consists of an ASCII file whose first line is a
description of the catalog, and subsequent lines should begin with a
valid file descriptor, followed by either a blank space, or a carriage return
character.
For exemple, 
\begin{verbatim}
##### This is a catalog ####
foo.fits
bar.fits[only]
bar/foo.fits xyz
\end{verbatim}
is a valid catalog. The actual name of a catalog should contain the
{\tt .cat} extension.
Every time a {\tt .cat} extension is specified on a
parameter, it is also possible to process directly one single fits
file, replacing the {\tt xyz.cat} parameter by a {\tt tuv.fits} file.

The uppercase letters refer to already existing FCLASSes. The {\tt
\_flat\_ } token is a short-hand for {\tt \_CON\_ } (spectral
channel), {\tt \_DOME\_} or {\tt \_SKY\_} (photometric channel), that
is, and exposure from 
which a flat-field, or a high-frequency flat-field can be computed.
The {\tt almost\_} prefix indicates that it is not yet the final stage
of preprocessing for photometric and red channel, but it is for the
blue channel.

The {\tt -in} parameter is the input file or list of files that are to
be processed. They are open in read-only mode, as well as the
auxilliary files specified by the options {\tt -bias}, {\tt -dark},
{\tt -flat}. Nothe that the {\tt -bias}, {\tt -dark} and
{\tt -flat} parameters are optional.
The {\tt -out} parameter is the output file name or list of file
names. In the case where an input and an output catalog are specified,
there is a one to one correspondance between the input and output file
names. That is, file name in input line $n$ will be processed and
saved with the output name specified at line $n$ of the output
catalog.
The special and optional {\tt -stackout} parameter repesents the
sigma-clipped mean of the files described by the {\tt -out}
parameter. For cosmic suppression, the number of processed files
should be at least 3. In the case of {\verb PRE_DARK_FRAME.fits }
generation, all input files should have the same exposure time.

The {\tt -bias} argument should be the result of either one of the
{\tt -out} files or the {\tt -stackout} file generated by {\tt
preprocess\_bias}. The same holds respectively for {\tt -dark}, {\tt
-flat} and the output of {\tt preprocess\_dark}, {\tt preprocess\_flat}.
Those parameters may be omitted in the {\tt preprocess} --- in that
case, no bias, dark, of flat-field correction is applied.

\end{subsection}

\begin{subsection}{The output image format}
The output is in fits format, with one extension named {\tt [image]}
witch contains the pixel values, and one extension names {\tt
[variance]} which contains the variance of the pixel values of the
[image] extension.

In order to open an image, you have to specify the {\tt '[image]'}
extension on the fits file on the command line. Do not forget to
enclose the name with single quotes.

If you cannot open extensions with your programs, you have to convert
your image in a no-extension fits using
\begin{verbatim}
> copy_image -in 'myimage.fits[image]' -out myimage_simple.fits
\end{verbatim}
and then working with {\verb myimage_simple.fits }

Alternatively, if you want an automatic image name conversion in your
source code, you may call the 
\begin{verbatim} char* ut_open_check_name(char* name);
\end{verbatim}
routine from the library (see the corresponding chapter for developpers).

\end{subsection}

\begin{subsection}{Current limitations of the preprocessing code}

The list of the code limitations contains both things that will be
improved in the future, and a list of code features, which are listed
for the knoiwledge

\begin{itemize}
\item Raster images are glued together, even if it is not a contiguous
image. 
OTcom raster images get 2 lines with bad data in the middle
columns of the exposure. Those lines are removed at the preprocessing
stage.
\item Most of detcom raster images will not contain an overscan
region. In that case, an estimate of the offset is performed from the
data region itself. This of course only senseful for exposures where
most of the pixels measure zero, like for instance an arc exposure.
This behaviour is useful for the focus exposures. Other detcom raster
exposures will have an uncontrolled zero substracted to them...
\item The preprocessing does not know how to handle a DETCOM binned image,
or a DETCOM raster image with overscan. In addition, in case of a DETCOM
raster, the CCDSEC fits keyword is wrong from online, and is not
corrected.
\item The odd-even correction is always applied, even in cases where
it will not help improving the noise.
\item The {\tt -stackout} option for {\tt preprocess\_dark} was not
fully tested, but is believed to work.
\item The {\tt preprocess\_flat} program assumes a photometric
exposure. No high-frequency flat-field is performed.
\item The {\tt split\_image} program does not smartly guess the
multi-filter parts, and the values are hardcoded from theoretical
regions, and were not updated with respect to real data.
\item No defringing is performed.
\item No linearity correction is applied.
\item No cosmetic map is applied
\item If the fits headers do not contain accurate inverse gain
numbers, the scaling of the image between left and right side will be
random.
\item The documentation is not finished.
\end{itemize}

\end{subsection}

\begin{subsection}{Other executables}

In addition to standard preprocessing, the following executables can
be found:

\begin{subsubsection}{Analysis tools}

\begin{itemize}
\item {\tt rootify} produces histograms in the ROOT\footnote{\tt http://root.cern.ch} package format.
\item {\tt rootify\_data} produces other histograms in the ROOT package
format.
\item {\tt rootify\_fft} produces specific histograms for FFT analysis.
\item {\tt rootify\_tree} produces data to help the decision of the
data quality program (still to be written).
\item {\tt stack\_bichip\_gain} produces a map where the gain estimate
is produced for each pixel, and has to be analysed with analysis tools.
\item {\tt fft\_image} compute a fourier-transform of the image line by
line.
\item {\tt quick\_look} prints basic statistic analysis of the image
\end{itemize}
\end{subsubsection}

\begin{subsubsection}{Miscellanous tools}

\begin{itemize}
\item {\tt rickify} produces a raw-image in standard detcom output
format out of a file coming from Rick Kessler's simulation.
\item {\tt add\_image} adds and scales 2 images in order to produce a
3rd one.
\item {\tt copy\_image} makes a simple image copy. This can be useful
for instance if you want to extract only 1 extension from a fits file.
\item {\tt copy\_image\_slice} makes image copy using the partial
loading of images mechanism, and is here only for benchmarking.
\item {\tt test\_extension} is a test program for the IFU library, to
see
how it handles fits file extensions.
\item {\tt wr\_desc\_cat} is a clone of the gentool wr\_desc, but with the
following modifications : it only accepts 2D frames, but it handles
catalog as input and output.
\end{itemize}

\end{subsubsection}

\begin{subsubsection}{Partial preprocessing}

The executables here perform only partial tasks for the preprocessing.
Some of them may be not up-to-date with respect to the rest of the
preprocessing code, as they do not belong to the standard testing procedure.
\begin{itemize}
\item {\tt bichip\_build} builds an image in detcom format out of a raw
exposure which may be either in detcom or otcom format.
%\item {\tt odd\_even\_correct} suppresses the odd-even effect.
\item {\tt bichip\_variance} builds the variance frame and fills it
with the readout noise computed from the overscan.
\item {\tt substract\_overscan} substract the offset computed from overscan.
\item {\tt bichip\_overscan} substract the offset computed from
overscan for the 2 preamplifiers.
\item {\tt stack\_bias} makes a sigma-clipping estimate of a frame out
of a stack using the variance stored in the variance frame assuming
gaussian noise, or
optionnaly computing and using the spread variance. It can be used on
non-bias frames, but 
the statistical output will be wrong in that case, as the underlying
statistics are not the good one (i.e. mix of poisson + gaussian
noise).
\item {\tt stack\_bichip} performes the same as {\tt stack\_image}, but
for the 2 preamplifiers.
\item {\tt bichip\_sub\_bias} substracts a bias frame.
\item {\tt bichip\_assemble} assembles the 2 sub-images from the 2
preamplifiers and scales them in order to get only 1 image.
\item {\tt add\_poisson} adds the poisson noise to the variance of the
image
\item {\tt substract\_dark} removes a dark frame.
\item {\tt build\_flat} only normalizes an image in order to produce a
flat frame.
\item {\tt substract\_flat} applies a flat frame.
\end{itemize}
\end{subsubsection}

\end{subsection}

